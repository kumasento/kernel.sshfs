\section{File System Implementation}

\subsection{File System Type}
It's required to provide a \lstinline{file_system_type} parameter if you want to register your file system.\\

\begin{lstlisting}
static struct file_system_type sshfs_fs_type = {
    .name     = "kernelsshfs",
    .mount    = sshfs_mount,
    .kill_sb  = sshfs_kill_sb,
    .fs_flags = FS_USERNS_MOUNT, // what for?
};

struct dentry *sshfs_mount(struct file_system_type *fs_type,
                           int flags,
                           const char *dev_name,
                           void *data)
{
    return mount_nodev(fs_type, flags, data, sshfs_fill_super);
}

static void sshfs_kill_sb(struct super_block *sb)
{
    kill_litter_super(sb);
}
\end{lstlisting}

\subsection{Super Block}

Look at the \lstinline{sshfs_mount} function, which will do \lstinline{mount_nodev}. \lstinline{mount_nodev} denotes our file system does not need a backend block device. And \lstinline{mount_nodev} need a function \lstinline{sshfs_fill_super} to fulfill the properties of the super block of our file system.

It's the definition of the \emph{super block operations}. The functions are provided by default Linux VFS super block operations.
\begin{lstlisting}
static const struct super_operations sshfs_ops = {
    .statfs       = simple_statfs,
    .drop_inode   = generic_delete_inode,
    .show_options = generic_show_options,
};
\end{lstlisting}

Here the \lstinline{sshfs_get_inode} will not be covered in this document, you could easily find it in the \lstinline{sshfs_inode.c} file. So the basic operations \lstinline{sshfs_fill_super} does are just setting:
\begin{enumerate}
\item \lstinline{s_op}
\item \lstinline{s_root} 
\item \lstinline{s_magic}
\end{enumerate}
\begin{lstlisting}
int sshfs_fill_super(struct super_block *sb, void *data, 
	int silent)
{
    sb->s_blocksize      = PAGE_CACHE_SIZE;
    sb->s_blocksize_bits = PAGE_CACHE_SHIFT;
    sb->s_magic          = SSHFS_MAGIC;
    sb->s_op             = &sshfs_ops;

    inode = sshfs_get_inode(sb, NULL, 
		S_IFDIR|SSHFS_DEFAULT_MODE, 0);

    sb->s_root = d_make_root(inode);
	...
}
\end{lstlisting}

\subsection{File and Inode Operations}

In order to satisfy our "virtual environment" requirement, we need to:
\begin{enumerate}
\item build \lstinline{inode} both on host and remote.
\item only do file operations on remote.
\item querying the host-only inodes.
\end{enumerate}

And the code definition of inode operations, and only the first 4 operations are self-defined. All the other operations are default operations which means they will not visit the remote information.

\begin{lstlisting}[title=\bfseries inode operations]
static const struct inode_operations 
	sshfs_dir_inode_operations = {
    .mknod  = sshfs_mknod,
    .mkdir  = sshfs_mkdir,
    .rmdir  = sshfs_rmdir,
    .create = sshfs_create,
    .lookup = simple_lookup,
    .link   = simple_link,
    .unlink = simple_unlink,
    .rename = simple_rename,
};
\end{lstlisting}

File operations are much more simplified, only \lstinline{read}, \lstinline{write} and \lstinline{llseek} are defined.

\begin{lstlisting}[title=\bfseries file operations]
const struct file_operations 
	sshfs_file_operations = {
    .read    = sshfs_file_read,
    .write   = sshfs_file_write,
    .llseek  = sshfs_llseek,
    .fsync   = noop_fsync,
};
\end{lstlisting}

All the self-defined operations are based on one function, \lstinline{netlink_send_command}, to send the command message like "read dir1/file1" to the netlink module. I'll cover this in the next section.


