\section{SSH Daemon}

The SSH daemon is just like a bridge, which lies between the kernel mode \lstinline{netlink} module, and the user mode \lstinline{libssh} library. 

\subsection{Communicate with Netlink}
Kernel mode netlink module could only response once SSH daemon send message to it. And SSH send message function follows these steps:
\begin{enumerate}
\item Initialize \lstinline{struct nlmsghdr} by assigning \lstinline{nlmsg_len}, \lstinline{nlmsg_pid} and \lstinline{nlmsg_flags}.
\item Initialize message itself. Send the message and stall.
\item Once received the message from the kernel, do the parsing work to get what to do next.
\end{enumerate}
\newpage
\begin{lstlisting}[title=\bfseries Communicate with Netlink]
char * send_recv_netlink_portal(char *msg_str, int msg_len)
{
    nlh = (struct nlmsghdr *) malloc(NLMSG_SPACE(MAX_PAYLOAD));
    memset(nlh, 0, NLMSG_SPACE(MAX_PAYLOAD));
    nlh->nlmsg_len   = NLMSG_SPACE(MAX_PAYLOAD);
    nlh->nlmsg_pid   = getpid();
    nlh->nlmsg_flags = 0;

    strncpy(NLMSG_DATA(nlh), msg_str, msg_len);

    iov.iov_base    = (void *)nlh;
    iov.iov_len     = nlh->nlmsg_len;

    msg.msg_name    = (void *)&dest_addr;
    msg.msg_namelen = sizeof(dest_addr);
    msg.msg_iov     = &iov;
    msg.msg_iovlen  = 1;

    sendmsg(sock_fd, &msg, 0);
	\\ will stall here
    recvmsg(sock_fd, &msg, 0);

    return (char *)NLMSG_DATA(nlh);
}
\end{lstlisting}

There's a infinite loop in the \lstinline{main()} function of SSH daemon:

\begin{lstlisting}[title=\bfseries Main loop]
while (1)
{
    char *recv =
        send_recv_netlink_portal("READY", strlen("READY"));
    if (!strcmp(recv, "ERROR"))
        ...        
    else if (!strcmp(recv, "NOTHING"))
        sleep(1);
    else
    {
        struct iovec *iov = 
            sftp_execute(ssh_s, sftp_s, recv);
        ...
        recv = 
            send_recv_netlink_portal("DONE", strlen("DONE"));
        if (strcmp(recv, "OK"))
            exit(1);
        sleep(1);
    }
}
\end{lstlisting}

It's clear that this daemon will first send \lstinline{READY}, and do different tasks by specifying the different responses.
If SSH daemon is sure that it'll do some work on the remote machine, it'll call \lstinline{sftp_execute}.

\subsection{Communicate with SSH Library}

There's no need to show some details about the libSSH library, so here I'll just take one function for instance: \lstinline{sftp_execute_mkdir}. This function will be called only if the command sent by netlink module is something like: \lstinline{mkdir dir1}. 

\begin{lstlisting}[title=\bfseries Remote Mkdir]
int sftp_execute_mkdir(ssh_session ssh, 
                       sftp_session sftp, 
                       char *dirname)
{
    int rc;

    rc = sftp_mkdir(sftp, dirname, S_IRWXU);
    if (rc != SSH_OK)
    {
        if (sftp_get_error(sftp) != SSH_FX_FILE_ALREADY_EXISTS)
        {
            fprintf(stderr, "Can't create directory: %s\n",
                    ssh_get_error(ssh));
            return rc;
        }
    }

    return rc;
}
\end{lstlisting}

As you can see, the key function is \lstinline{sftp_mkdir}. 

Another example is \lstinline{sftp_execute_read}:

\begin{lstlisting}[title=\bfseries Remote Read]
int sftp_execute_read(ssh_session ssh, 
                      sftp_session sftp, 
                      char *params, char **msg)
{   ....
    file = sftp_open(sftp, filename, access_type, 0);
    sftp_seek(file, offset);
    for (;;)
    {
        nbytes = sftp_read(file, buffer, sizeof(buffer));
        ....
        if (nbytes == 0) break;
        else if (nbytes < 0) return SSH_ERROR;

        int buflen = strlen(buffer);
        int msglen = (*msg != NULL) ? strlen(*msg) : 0;
        msglen += buflen;
        msglen ++;
        ....
    }
    rc = sftp_close(file);
    return rc;
}
\end{lstlisting}

The data received from remote operation \lstinline{sftp_read} will be inserted into local buffer, and send back to netlink once the transfer is finished. 

After the netlink received the message, it will copy the content to the \lstinline{read()} API's buffer.



